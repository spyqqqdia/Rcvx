\section{Hessian}

The Hessian of $f$ at a point $x$, denoted $B=H(f)(x)=\grad^2f(x)$, 
is the second derivative of
a function $f:\bbR^n\to\bbR$ at the point $x$ and hence a bilinear map
$$
B:\bbR^n\times\bbR^n\to\bbR.
$$
it is usually identified with the matrix
$$
H_{ij}=\grad^2f(x)(e_i,e_j)=\frac{\partial^2f}{\partial x_i\partial x_j}(x)
$$
and with this identification we have
$$
B(u,v)=u^THv.
$$
Let $f:\bbR^n\to\bbR$ be a $C^2$-function and $\ol f(u)=f(x_0+Fu)$, where  $x_0\in\bbR^n$
and $F:\bbR^m\to\bbR^n$ is a linear map, i.e. $F\in Mat_{n\times m}$.

We want to compute the gradient and Hessian of $h$ at any point $u$ from those of $f$.
To get these do a second order Taylor expansion of $f$ about $x$:
$$
f(x+h)=f(x)+h^Tg+h^THh+o(\norm h^2),
$$
where $g=\grad f(x)$ and $H=\grad^2f(x)$ are uniquely determined by the fact that
the residual is $o(\norm h^2)$. Applying this to the point $x=x_0+Fu$ this implies
that
%
\begin{align*}
\ol f(u+h)&=f(x_0+Fu+Fh)
\\&=
f(x_0+Fu)+(Fh)^Tg+(Fh)^THFh+o(\norm{Fh}^2).
\end{align*}
%
Since $o(\norm{Fh}^2)$ is $o(\norm{h}^2)$ we conclude from this that
$$
\grad\ol f(u)=F^Tg\q\text{and}\q \grad^2\ol f(u)=F^THF,
$$
or, more explicitly
%
\begin{equation}
\label{diff_lin_trans}
\grad\ol f(u)=F^T\grad f(x_0+Fu)
\q\text{and}\q
\grad^2\ol f(u)=F^T\grad^2 f(x_0+Fu)F.
\end{equation}
%
With a similar approach we can compute the Hessian of a composition $g(f(x))$, where
here $g:\bbR\to\bbR$ is a scalar function of one variable (more general $g$ are much harder
to handle and we do not need them).
Indeed, set
$$
y=f(x),\q\grad=\grad f(x)\q H=\grad^2f(x)\text{and}\q k=h^T\grad+\tfrac{1}{2}h^THh
$$
and use second order Taylorapproximations on $f$ at
the point $x$ and $g$ at the point $y=f(x)$ to obtain:
%
\begin{align*}
g(f(x+h)))&=g\left(f(x)+h^T\grad+\tfrac{1}{2}h^THh\right)
\\&=
g(y+k)=g(y)+g'(y)+\tfrac{1}{2}g''(y)k^2+o(k^2)
\\&=
g(y)+g'(y)\left(h^T\grad+\tfrac{1}{2}h^THh\right)+
\frac{1}{2}g''(y)\left(h^T\grad+\tfrac{1}{2}h^THh\right)^2+o(\norm h^2).
\end{align*}
%
Here we have used that $o(k^2)=o(\norm h^2)$.
Collect terms of first and second order in $h$ together and sticking
all terms of higher order into the residual $o(\norm h^2)$. Note that
the squared term contributes no first order terms and only one second order term,
this being the term
$$
(h^T\grad)^2=
(h^T\grad)(h^T\grad)=
(h^T\grad)(h^T\grad)^T=
h^T(\grad\grad^T)h.
$$
We obtain
$$
g(f(x+h)))=g(y)+h^T[g'(y)\grad]+
\frac{1}{2}h^T\left[
g'(y)H+g''(y)\grad\grad^T
\right]h+o(\norm h^2).
$$
and from this we can read off that
%
\begin{align}
\label{taylor_of_composition}
\grad(g\circ f)(x)&=g'(f(x))\grad f(x),\q\text{and}\\
H(g\circ f)(x)&=g'(f(x))H+g''(f(x))\grad f(x)\grad f(x)^T
\end{align}
%
Note that here $d=\grad f(x)$ is viewed as a column vector and so $\grad f(x)\grad f(x)^T$
is the \textit{outer product}
$$
\grad f(x)\grad f(x)^T=dd^T=(d_id_j)_{ij}.
$$
 
    